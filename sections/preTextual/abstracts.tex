% resumo em português
\setlength{\absparsep}{18pt} % ajusta o espaçamento dos parágrafos do resumo
\begin{resumo}
   O uso de redes de sensores na indústria vem crescendo de forma significativa com o passar dos anos, principalmente o uso de redes de sensores sem fio industriais (RSSFIs), devido ás suas vantagens em relação ás redes cabeadas. Entretanto, essas redes carregam um conjunto de desafios, o principal deles é a falta de confiabilidade no canal de comunicação, devido a fontes de interferência e o alto nível de atenuação. Além disso, as RSSFIs apresentam variações nas características do canal ao longo do tempo. Fornecer uma boa qualidade de serviço durante sua operação é de suma importância para implementar uma RSSFI. Este trabalho tem como objetivo dar continuidade a dois trabalhos anteriores que visam melhorar a confiabilidade na comunicação das RSSFIs por meio do uso de diversidade de modulação adaptativa, onde os nós conseguem estimar a qualidade do enlace e decidir a melhor modulação a ser usada no momento. Para isso, foi usado um conjunto de dados obtidos num experimento que utiliza a emenda 'g' do padrão IEEE 802.15.4, padrão esse destinado a redes sem fio pessoais de baixa potência. No experimento chegou-se á conclusão de que nenhuma modulação fornece por si só uma alta taxa de entrega de pacotes na camada de aplicação para todos os nós durante todo o tempo. Esse resultado serviu de base para o desenvolvimento de um mecanismo de modulação adaptativa, que utiliza as três modulações baseadas em uma probabilidade determinada por uma equação e utiliza os pacotes ACK recebidos no transmissor para estimar a qualidade do enlace. Como primeira contribuição, foi proposta uma atualização no mecanismo, que adiciona uma estimação de qualidade de enlace no receptor. A segunda contribuição do trabalho foi juntar os dois estimadores e criar um estimador híbrido, que utiliza das informações obtidas pelos dois estimadores,  criando um mais complexo. Os resultados obtidos por meio de simulações indicam uma melhoria na taxa de entrega de pacotes (PDR) em alguns nós.
   
\vspace{\onelineskip}

  \textbf{Palavras-chaves}: Redes de sensores sem fio industriais, diversidade de modulação, mecanismos de estimação de qualidade de enlace. 
\end{resumo}

% resumo em inglês
\begin{resumo}[Abstract]
  \begin{otherlanguage*}{english}
    The use of sensor networks in the industry has grown significantly over the years, mainly the use of Wireless Sensor Networks (WSNs), due to their advantages over wired networks. However, these networks carry a series of challenges, the main one being the lack of reliability in the communication channel, due to sources of interference and the high level of attenuation. In addition, WSNs are susceptible to variations in channel characteristics over time. Providing a good quality of service during their operation is of paramount importance for implementing WSNs. This work is based on two previous works that aim to improve the reliability of the communication of WSNs through the use of adaptive modulation diversity, where the nodes are able to estimate the quality of the link and decide the best modulation to be used at each moment. For this, we used a dataset obtained in an experiment that uses the 'g' amendment of the IEEE 802.15.4 standard, which is intended for low-power wireless networks. In the experiment it was concluded that no modulation alone provides a high packet delivery ratio at the application layer. This result served as a basis for the development of an adaptive modulation mechanism, which uses the three modulations based on a probability determined by an equation and uses the ACK packets received at the transmitter to estimate the quality of the link. As a first contribution, an update on the mechanism was proposed, in which it adds an estimate of link quality at the receiver. The second and main contribution of the work, was to join the two estimators and create a hybrid estimator, which uses the information obtained by the two estimators and creates a more complex one. The results obtained through simulations indicate an improvement in the packet delivery rate (PDR) in some nodes.

    \vspace{\onelineskip}

    \noindent
    \textbf{Key-words}: Industrial wireless sensor networks, modulation diversity, link quality estimation mechanisms.
  \end{otherlanguage*}
\end{resumo}

% resumo em francês 
% \begin{resumo}[Résumé]
%   \begin{otherlanguage*}{french}
%     Il s'agit d'un résumé en français.

%     \textbf{Mots-clés}: latex. abntex. publication de textes.
%   \end{otherlanguage*}
% \end{resumo}

% % resumo em espanhol
% \begin{resumo}[Resumen]
%   \begin{otherlanguage*}{spanish}
%     Este es el resumen en español.

%     \textbf{Palabras clave}: latex. abntex. publicación de textos.
%   \end{otherlanguage*}
% \end{resumo}
